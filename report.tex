\documentclass[a4paper,12pt]{article}
\usepackage{xcolor}
\usepackage{amsmath,amsfonts,amssymb}
\usepackage{geometry}
\usepackage{fancyhdr}
\usepackage{graphicx}
\usepackage{titlesec}
\usepackage{tikz}
\usepackage{booktabs}
\usepackage{array}
\usetikzlibrary{shadows}
\usepackage{tcolorbox}
\usepackage{float}
\usepackage{lipsum}
\usepackage{mdframed}
\usepackage{pagecolor}
\usepackage{mathpazo}   % Palatino font (serif)
\usepackage{microtype}  % Better typography

% Page background color
\pagecolor{gray!10!white}

% Geometry settings
\geometry{margin=0.5in}
\pagestyle{fancy}
\fancyhf{}

% Fancy header and footer
\fancyhead[C]{\textbf{\color{blue!80}CS348 Assignment-3}}
\fancyhead[R]{\color{blue!80}Saksham Rathi}
\fancyfoot[C]{\thepage}

% Custom Section Color and Format with Sans-serif font
\titleformat{\section}
{\sffamily\color{purple!90!black}\normalfont\Large\bfseries}
{\thesection}{1em}{}

% Custom subsection format
\titleformat{\subsection}
{\sffamily\color{cyan!80!black}\normalfont\large\bfseries}
{\thesubsection}{1em}{}

% Stylish Title with TikZ (Enhanced with gradient)
\newcommand{\cooltitle}[1]{%
  \begin{tikzpicture}
    \node[fill=blue!20,rounded corners=10pt,inner sep=12pt, drop shadow, top color=blue!50, bottom color=blue!30] (box)
    {\Huge \bfseries \color{black} #1};
  \end{tikzpicture}
}
\usepackage{float} % Add this package

\newenvironment{solution}[2][]{%
    \begin{mdframed}[linecolor=blue!70!black, linewidth=2pt, roundcorner=10pt, backgroundcolor=yellow!10!white, skipabove=12pt, skipbelow=12pt]%
        \textbf{\large #2}
        \par\noindent\rule{\textwidth}{0.4pt}
}{
    \end{mdframed}
}

% Document title
\title{\cooltitle{CS378 Lab-5 Report}}
\author{
    {\bf Saksham Rathi (22B1003)} \\ 
    Department of Computer Science, IIT Bombay \\ 
    \and
    {\bf Nandan Manjunath (22B0920)} \\ 
    Department of Computer Science, IIT Bombay
}
\date{}

\begin{document}
\maketitle
\begin{solution}{Part1: Theory}
    Let us prove this by induction.
    \subsection*{Case1 : number of links(k) = 1}
    Packet 1 is ahead and reaches $t_1=P/C_1$ where $C_1$ is the link's bandwidth. Packet 2 waits until Packet 1 is completely sent and then is sent. It takes time $P/C_1+P/C_1$, which is $t_2=2*P/C_1$. 

    In this case, $P/(t_2-t_1)$ is $C_1$, which is the correct bottleneck bandwidth. Therefore, the base case is true.

    \subsection*{Case2 : True for number  of links(k) then true for links(k+1)}
    Until it covers k links, let the time they take be $t_1$ and $t_2$, and bottleneck bandwidth be C which is $P/(t_2-t_1)$ (induction assumption). This bandwidth C is the minimum among all links k. Let  ${k+1}^{th}$ link bandwidth be $D$.
    \subsubsection*{Case2.1: If D is greater than or equal to C}
    Now new $t_1'$ will become $t_1+P/D$ and as $D>=C$, $t_1'$ will be less than $t_2$ therefore, by the time complete Packet 2 crosses link k, complete Packet 1 will cross  ${k+1}^{th}$ link also. So, Packet 2 doesn't wait, and the new $t_2'$ will be $t_2+P/D$. In this case, $P/(t_2'-t_1')$ is C, which matches the bottleneck bandwidth.
    \subsubsection*{Case2.2: Id D is less than C}
    Then, the bottleneck bandwidth will become D. The new $t_1'$ is $t_1+P/D$.
    and as $D<C$, $t_1'$ will be more than $t_2$ therefore Packet2 has to wait until Packet 1 is completely sent across  ${k+1}^{th}$ link. This waiting time is $t_1'-t_2$, and then it is sent. The new $t_2'$ is $t_2+$waiting time $+P/D$ which is $t_2+t_1'-t_2+P/D$. In this case, $P/(t_2'-t_1')$ is D, which matches the bottleneck bandwidth.

    By induction, we proved that bottleneck bandwidth can be calculated as $P/(t_2-t_1)$
\end{solution}


\end{document}
