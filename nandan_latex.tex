\documentclass{article}
\usepackage{graphicx} % Required for inserting images

\title{cs378-lab5}
\author{Nandanmanjunath I}
\date{October 2024}

\begin{document}

\maketitle

\section{Part1: Theory}

Let us prove this by induction.

\subsection{Case1 : number of links(k) = 1}
In this case, both packets start at time t=0. Packet 1 is ahead and reaches $t_1=P/C_1$ where $C_1$ is the link's bandwidth. Packet 2 waits until Packet 1 is completely sent and then starts sending. It takes time $P/C_1+P/C_1$, which is $t_2=2*P/C_1$. 

In this case, $P/t_2-t_1$ is $C_1$, which is the correct bottleneck bandwidth. Therefore, the base case is true.

\subsection{Case2 : True for number  of links(k) then true for links(k+1)}
Until it covers k links, let the time they take be $t_1$ and $t_2$, and bottleneck bandwidth be C which is $P/t_2-t_1$. This bandwidth C is the minimum among all links k. Let  ${k+1}^{th}$ link bandwidth be $D$.
\subsubsection{Case2.1: If D is greater than or equal to C}
Now new $t_1'$ will become $t_1+P/D$ and as $D>=C$, $t_1'$ will be less than $t_2$ therefore, by the time complete Packet 2 crosses link k, complete Packet 1 will cross  ${k+1}^{th}$ link also. So, Packet 2 doesn't wait, and the new $t_2'$ will be $t_2+P/D$. In this case, $P/t_2'-t_1'$ is C, which matches the bottleneck bandwidth.
\subsubsection{Case2.2: Id D is less than C}
Then, the bottleneck bandwidth will become D. The new $t_1'$ is $t_1+P/D$.
and as $D<C$, $t_1'$ will be more than $t_2$ therefore Packet2 has to wait until Packet 1 is completely sent across  ${k+1}^{th}$ link. This waiting time is $t_1'-t_2$, and then it is sent. The new $t_2'$ is $t_2+$waiting time $+P/D$ which is $t_2+t_1'-t_2+P/D$. In this case, $P/t_2'-t_1'$ is D, which matches the bottleneck bandwidth.

By induction, I proved that bottleneck bandwidth can be calculated as $P/t_2-t_1$

\end{document}